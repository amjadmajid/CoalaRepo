% Copyright (c) 2013 Joost van Zwieten
%
% Permission is hereby granted, free of charge, to any person obtaining a copy
% of this software and associated documentation files (the "Software"), to deal
% in the Software without restriction, including without limitation the rights
% to use, copy, modify, merge, publish, distribute, sublicense, and/or sell
% copies of the Software, and to permit persons to whom the Software is
% furnished to do so, subject to the following conditions:
%
% The above copyright notice and this permission notice shall be included in
% all copies or substantial portions of the Software.
%
% THE SOFTWARE IS PROVIDED "AS IS", WITHOUT WARRANTY OF ANY KIND, EXPRESS OR
% IMPLIED, INCLUDING BUT NOT LIMITED TO THE WARRANTIES OF MERCHANTABILITY,
% FITNESS FOR A PARTICULAR PURPOSE AND NONINFRINGEMENT. IN NO EVENT SHALL THE
% AUTHORS OR COPYRIGHT HOLDERS BE LIABLE FOR ANY CLAIM, DAMAGES OR OTHER
% LIABILITY, WHETHER IN AN ACTION OF CONTRACT, TORT OR OTHERWISE, ARISING FROM,
% OUT OF OR IN CONNECTION WITH THE SOFTWARE OR THE USE OR OTHER DEALINGS IN
% THE SOFTWARE.
%
\documentclass{tudelftposter}

% optional, makes QR code clickable
% \usepackage[hidelinks,implicit=false,bookmarks=false]{hyperref}

\usepackage{amsmath}

\newcommand{\sys}{IPE}
\newcommand{\sysfull}{Intermittently-Powered Executor}

\title{\sys{}: \sysfull{}}

\addauthor{\large Amjad Y. Majid}

\addauthornote{mail}{a.y.majid@tudelft.nl}
\addauthornote{uni}{Delft University of Technology, Delft, NL}



% \addfootimage(c:right column.center)[Delft Institute of Applied Mathematics]{tudelft}
% \addfootqrcode(l:left column.left)[latex-poster-class repository]{https://github.com/tudelft-diam-na/latex-poster-class}

\newlength{\textboxwidth}
\newlength{\textboxcolumnsep}\setlength{\textboxcolumnsep}{1em}

\setlength\arrayrulewidth{2pt}
\def\centeredhrulefill{\leaders\hbox{\rule[.5ex]{5pt}{\arrayrulewidth}}\hfill}

% source: http://tex.stackexchange.com/questions/23100/looking-for-an-ignorespacesandpars
\makeatletter
\def\ignorespacesandpars{%
  \begingroup
  \catcode13=10
  \@ifnextchar\par
    {\endgroup\expandafter\ignorespacesandpars\@gobble}%
    {\endgroup}%
}
\makeatother

\def\textbox(#1) at (#2) [#3] <#4,#5> title #6{%
  \def\textboxname{#1}%
  \def\textboxpos{#2}%
  \def\textboxalign{#3}%
  \def\textboxncolumns{#4}%
  \setlength{\textboxwidth}{#5}%
  \def\textboxtitle{#6}%
%
  \ifdefstring{\textboxncolumns}{1}{%
    % single column
    \def\atbeginofminipage{%
      \vspace{.5\baselineskip}%
      \par\noindent\ignorespacesandpars}%
    \def\atendofminipage{}%
  }{%
    % multiple columns
    \setlength{\textboxwidth}{\textboxncolumns\textboxwidth}%
    \addtolength{\textboxwidth}{\textboxncolumns\textboxcolumnsep}%
    \addtolength{\textboxwidth}{-\textboxcolumnsep}%
    \def\atbeginofminipage{%
      \setlength{\columnsep}{1em}%
      \begin{multicols}{\textboxncolumns}%
        \par\noindent\ignorespacesandpars}%
    \def\atendofminipage{%
      \end{multicols}}%
  }%
%
  \node[inner sep=0pt,outer sep=0cm] (#1) at (#2) [#3] \bgroup%
    \begin{minipage}{\textboxwidth}%
      \setlength{\parskip}{0pt}%
      \setlength{\parindent}{1em}%
      \par\noindent%
      {% title
        \color{tudcyan}\rm\scshape%
        \centeredhrulefill\hspace{1em}%
        \textboxtitle%
        \hspace{1em}\centeredhrulefill}%
      \atbeginofminipage}%
\def\endtextbox{%
      \atendofminipage%
    \end{minipage}%
  \egroup;%
}

\begin{document}
\begin{tikzpicture}[remember picture,overlay,line width=\arrayrulewidth]

\begin{textbox}(introduction)
at (body.north west) [below right] <2,24cm>
title {Introduction}

Enabling battery-free devices is a mandatory step towards an environment friendly Internet of Things (IoT). However, removing the batteries requires IoT to operate on an harvested power supply. This make sustaining long computation very challenging. Correspondingly, there are two approaches to enable long-running computations on intermittently-powered IoT: (i) checkpointing, where the volatile state of a program is frequently saved to the non-volatile memory, and (ii) task-based approach, where a programmer splits the code into small idempotent modules. Results show that task-based approach performs better than checkpointing. However, the task-based model is static approach that does not take advantage of variation in the amount of the harvested energy or adapts to different harvester configuration. Therefore, we introduce \sysfull{} which pushes the boundaries of the task-based model by enabling an application to enlarge its task size, on the fly, when there is redundant energy. 

% \sysfull{} (\sys{}) is runtime library that facilitates tasks navigation and preserves data/memory consistency of IPDs. \sys{} aims at reducing energy consumption and applications execution time. It optimizes the commit rate (saving tasks contexts into the non-volatile memory), subject to the number of power interrupts and completed tasks.

\end{textbox}

\begin{textbox}(spam)
at (body.north east) [below left] <1,24cm>
title {Spam}

  \begin{itemize}
    \item Nulla malesuada porttitor diam. Donec felis erat, congue non, volutpat
    at, tincidunt tristique, libero. Vivamus viverra fermentum felis. Donec
    nonummy pellentesque ante.
    \item Phasellus adipiscing semper elit. Proin fermentum massa ac quam. Sed
    diam turpis, molestie vitae, placerat a, molestie nec, leo. Maecenas
    lacinia.
    \item Nam ipsum ligula, eleifend at, accumsan nec, suscipit a, ipsum.  Morbi
    blandit ligula feugiat magna. Nunc eleifend consequat lorem. Sed lacinia
    nulla vitae enim.
  \end{itemize}

\end{textbox}

\begin{textbox}(initresults)
at ([shift={(0pt,-29cm)}]body.north west) [below right] <1,40cm>
title {preliminary results}

  \centering
    \adjincludegraphics[width=.47\columnwidth]{./fig/envAwareness}

  \vspace{-1cm}

  \justify
  \noindent\textbf{App execution time, on a varying distance from the reader}

  \centering
      \adjincludegraphics[width=.47\columnwidth]{./fig/fram_write}

    \vspace{-1cm}
   \textbf{Energy cost of FRAM and RAM write access}

\end{textbox}

\begin{textbox}(foobar)
at (body.south west) [above right] <1,35cm>
title {Foo \& Bar}%

  Nam dui ligula, fringilla a, euismod sodales, sollicitudin vel, wisi. Morbi
  auctor lorem non justo. Nam lacus libero, pretium at, lobortis vitae, ultricies
  et, tellus. Donec aliquet, tortor sed accumsan bibendum, erat ligula aliquet
  magna, vitae ornare odio metus a mi nascetur ridiculus mus.
  \begin{equation}
    \int_\Omega v_i \left(
      u_{i,t} + u_j u_{i,j} + p_{,i} - R^{-1} u_{i,jj}
    \right) =
    \int_\Omega v_i f_i.
  \end{equation}
  Morbi ac orci et nisl hendrerit mollis.  Suspendisse ut massa. Cras nec ante.
  Pellentesque a nulla.  Cum sociis natoque penatibus et magnis dis parturient
  montes, nascetur ridiculus mus. Aliquam tincidunt urna. Nulla ullamcorper
  vestibulum turpis. Pellentesque cursus luctus mauris.

\end{textbox}

\begin{textbox}(conclusion)
at (body.south east) [above left] <1,35cm>
title {Conclusion}%

  Suspendisse vel felis. Ut lorem lorem, interdum eu, tincidunt sit amet, laoreet
  vitae, arcu. Aenean faucibus pede eu ante.
  \begin{itemize}
    \item Quisque vehicula, urna sed ultricies auctor, pede lorem egestas dui, et
    convallis elit erat sed nulla.
    \item Donec luctus.
    \item Curabitur et nunc.
  \end{itemize}
  Integer arcu est, nonummy in, fermentum faucibus, egestas vel, odio.

\end{textbox}

\node (fig1)
at ([shift={(0pt,-37cm)}]body.north east) [below left] {%
  \begin{tikzpicture}[x=5cm,y=5cm,line width=\arrayrulewidth]
    \draw (0,0) -- (4,0) node[anchor=north] {$x$};
    \draw (0,0) -- (0,3) node[anchor=east] {$f(x)$};
    \draw
      (0,0) node[anchor=north] {0}
      (1,0) node[anchor=north] {1}
      (2,0) node[anchor=north] {2}
      (3,0) node[anchor=north] {3};
    \draw [tudcyan] plot [smooth, tension=1]
      coordinates {(0,0) (1,2) (2,1) (3,3)};
  \end{tikzpicture}};
\node at ([shift={(-2ex,0)}]fig1.south west) [above left] {%
  \begin{minipage}{15cm}
    \itshape\raggedleft%
    {\rm\scshape Figure 1}\\
    \small%
    Morbi fringilla, wisi in dignissim interdum, justo lectus sagittis dui, et
    vehicula libero dui cursus dui
  \end{minipage}};
\draw[tudcyan] ([shift={(-1ex,0)}]fig1.south west)--([shift={(-1ex,0)}]fig1.north west);

\end{tikzpicture}
\end{document}
% vim: tw=80:ts=2:sts=2:sw=2:et:fdm=marker:fmr=[[[,]]]
