% Connecting the work to a real-world problem
Embedded devices in the Internet of Things (IoT) gain new deployment
opportunities when their hardware is free of batteries and powered by ambient
energy.
% The main problem that we are tackling
Energy harvested from the environment, however, is unpredictable
and can only power a device intermittently.
% The already proposed solution to the main problem
In the intermittent execution model, computation is frequently interrupted by
power failures, and to make forward progress, the system must save intermediate
program state into non-volatile memory.

% The problem of the proposed solutions 
Task-based programming models that execute at the granularity of
statically-defined tasks, offer an efficient alternative to checkpointing.
%
Existing task-based systems, however, fix their task size at compile time and
operate without any regard to the changing energy conditions at runtime.
%
When tasks are sized without accounting for available energy, state may be
persisted either more frequently than unnecessary or not frequently enough for
the program to make progress and terminate.
%
% Our main claim 
To address this challenge, we propose Coala, an \emph{adaptive} task-based
execution model that colesces or downscales static tasks at runtime in response
to changing energy availability.
% show off
% main benefits of Coala
Coala ensures that power failures cannot leave the program state in
non-volatile memory inconsistent using a specialized memory virtualization
mechanism.
%
Our evaluation on a real energy-harvesting device demonstrates the benefits of
adaptive task coalescing and downscaling, and highlights Coala's performance
improvements over a state-of-art task-based system.


