Enabling battery-free devices is a mandatory step towards an environment-friendly Internet of Things (IoT). However, removing the batteries requires IoT to operate on an harvested, ergo intermittent, power supply. This make sustaining long computation very challenging. Correspondingly, there are two approaches to enable long-running computations on intermittenly-powered IoT: (i) checkpointing, where the volatile state of a program is frequently saved to the non-volatile memory, and (ii) task-based approach, where a programmer splits the code into small idempotent modules. Results so far suggest that task-based approach performs better than checkpointing. However, the use of tasks requires that the energy to execute any task must not exceed the maximum limit of the energy buffer. This makes the code unefficient (when executed on a large energy buffer than indended) and potentially completely disables the program (in the opposite case). Therefore, we present \sys: a task-based execution runtime, enabling code portability by means of a new concept of task coalescing. If a code is written for a very small energy buffer, \sys is able to dynamically coalesce these tasks and constructs a bigger (virtual) task to improve the performance. \todo{Re-write the abstract once the paper is ready}{Brandon}