\TODO{Need to re-write this to make it consistent with the story after we have the story in place}
Enabling battery-free devices is a mandatory step towards the environment friendly IoT. However, removing these large energy reservoirs (the batteries) makes the devices operate on an intermittent power supply, which make sustaining any long computation very challenging. correspondingly, there are two proposed approaches to enable long-running computations on intermittenly powered devices (IPDs): (i) Checkpointing based approach, where the volatile state of a program is frequently saved to the non-volatile memory and; (ii)  Task based approach, where a programmer splits the code into small idempotent modules. The Task based approach shows a better performance than the checkpointing approach. However, the task based method requires that the energy to execute any task must not exceed the maximum limit of the energy buffer. This limitation makes this solution a customized one that suits a particular energy buffer size. 

VIPOS enables code portability by means of virtualizing the task based model. If a code is written for a very small energy buffer, VIPOS is able to dynamically merge these tasks and constructs a bigger virtual task to improve the performance, reduce energy consumption and provides mean for code portability. 
