Computation consistency on intermittently-powered devices  can be ensured by: (i) checkpointing, where the volatile state of a program is frequently saved to non-volatile memory, or (ii) via tasks, where a programmer splits the code into small idempotent sections. Results so far suggest that a task-based approach performs better than checkpointing. However, existing tasks-based approaches do not adapt to changing energy levels and storage. This makes the code inefficient (when executed on a larger energy buffer than intended) or can potentially completely disable the program (in the opposite case). Therefore, we present \sys: a task-based execution runtime, enabling code portability by means of a new concept of task coalescing and automatic compiler-supported task generation. If a code is written for a very small energy buffer, \sys is able to dynamically coalesce these tasks and constructs a bigger (virtual) task to improve execution times.