% Connecting the work to a real-world problem
%Intermittently powered computers promise a sustainable Internet of things by forgo the batteries and scarf the energy from the environment. 
Massively deployed Internet of Things (IoT) must be energy neutral. This can only be achieved when the IoT is batteryless and the energy to operate the IoT comes from ambient energy sources.
% The main problem that we are tackling
However, relying on the highly varying, unpredictable and intermittent harvested energy is at odds with the core assumption that program executions require stable and uninterrupted energy supply. 
% The already proposed solution to the main problem
The core programming model for the intermittently-powered computation is a discrete-like one that alternates between progressing and protecting the execution progress against power interrupts.
% The problem of the proposed solutions 
Generally, intermittent computation models can be classified into (i) checkpointing, where the volatile state of a program is frequently saved to non-volatile memory, or (ii) task-based programming, where a programmer splits the code into small idempotent sections. Despite the superiority of task-based systems over checkpointing, expressed in lower memory overhead, task-based programming is completely oblivious to the energy conditions. Simply, it cannot increase or decrease the size of the task on the fly (the moment of commit to the non-volatile memory) depending on how long the energy burst is. 
% Our main claim 
To address this challenge, this paper introduces \sys, an \emph{adaptive} task-based execution model that does not require any dedicated hardware to support task adaptability, which at the same time is \emph{efficient}.
% show off
\sys programs tune frequency of switching between computation progressing and program state protection based on the available energy, taking the burden of finding the right computation/protection period off the programmer's shoulders.
% main benefits of Coala
\sys defaults to progress on a multi-tasks scale, when energy permits (\emph{efficiency}), and on a sub-task scale when needed (\emph{guarantee}).
% I really hope it will support our claims!! haha
Results show that \sys is more efficient than the current state-of-art execution model by up to a factor of two.

% old PLDI abstract 
%Computation consistency on intermittently-powered devices can be ensured by: (i) checkpointing, where the volatile state of a program is frequently saved to non-volatile memory, or (ii) via tasks, where a programmer splits the code into small idempotent sections. Results so far suggest that a task-based approach performs better than checkpointing. However, existing tasks-based approaches do not adapt to changing energy levels and storage. This makes the code inefficient (when executed on a larger energy buffer than intended) or can potentially completely disable the program (in the opposite case). Therefore, we present \sys: a task-based execution runtime, enabling code portability by means of a new concept of task coalescing. If a code is written for a very small energy buffer, \sys is able to dynamically coalesce these tasks and constructs a bigger (virtual) task to improve execution times up to 4.5 times compared to static task code.
