Enabling battery-free devices is a mandatory step towards an environment-friendly Internet of Things (IoT). However, removing the batteries requires IoT to operate on a harvested, ergo intermittent, power supply. This makes sustaining long computations challenging. Basically, there are two approaches: (i) checkpointing, where the volatile state of a program is frequently saved to non-volatile memory, and (ii) task-based approach, where a programmer splits the code into small idempotent sections. Results so far suggest that a task-based approach performs better than checkpointing. However, the use of tasks requires that the energy to execute any task must not exceed the capacity of the energy buffer. This makes the code inefficient (when executed on a larger energy buffer than intended) and potentially completely disables the program (in the opposite case). Therefore, we present \sys: a task-based execution runtime, enabling code portability by means of a new concept of task coalescing and automatic compiler-supported task generation. If a code is written for a very small energy buffer, \sys is able to dynamically coalesce these tasks and constructs a bigger (virtual) task to improve the performance. \todo{Re-write the abstract once the paper is ready}{Brandon}