% Connecting the work to a real-world problem
Battery-free computers promise a sustainable Internet of things by forgo the batteries and scarf the energy from the environment. 
% The main problem that we are tackling
However, relying on the surrounding energy when it is available is at odds with current execution model of long-running computation. 
% The already proposed solution to the main problem
Scientists have proposed a discrete-like execution model that alternates between progressing the execution and protecting the execution progress, against power interrupts.
% The problem of the proposed solutions 
Generally, the proposed techniques can be classified as (i) checkpointing systems, where the volatile state of a program is frequently saved to non-volatile memory, or (ii) task-based systems, where a programmer splits the code into small idempotent sections. Despite the superiority of the task-based techniques, they suffer from a fixed overhead and they are completely blind to the energy conditions.   
% Our main claim 
In this paper we introduce \sys an \emph{efficient} and \emph{adaptive} task-based execution model.
% show off
\sys tunes its frequency of switching between progressing and protecting based on the available energy, but 
% One of the main functions of an OS is to abstract the hardware and to be less dependent on it. Coala does that too!
tt does not require hardware support to refine its estimation. 
% main benefits of Coala
\sys defaults to progress on a multi-tasks scale, when energy permits (\emph{efficiency}), and on a sub-task scale when needed {\emph{guarantee}}. 
% I really hope it will support our claims!! haha
Results show the \sys is more efficient than the current state-of-art execution model by X-times and more reliable. 

% old abstract 
%Computation consistency on intermittently-powered devices can be ensured by: (i) checkpointing, where the volatile state of a program is frequently saved to non-volatile memory, or (ii) via tasks, where a programmer splits the code into small idempotent sections. Results so far suggest that a task-based approach performs better than checkpointing. However, existing tasks-based approaches do not adapt to changing energy levels and storage. This makes the code inefficient (when executed on a larger energy buffer than intended) or can potentially completely disable the program (in the opposite case). Therefore, we present \sys: a task-based execution runtime, enabling code portability by means of a new concept of task coalescing. If a code is written for a very small energy buffer, \sys is able to dynamically coalesce these tasks and constructs a bigger (virtual) task to improve execution times up to 4.5 times compared to static task code.
