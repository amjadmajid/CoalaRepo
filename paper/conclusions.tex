This paper presented \sys: a new task-based runtime for intermittently-powered devices that does not require any hardware support. \sys core new features are: (i) automatic task-generation---removing programmer intervention in the program re-composition to the intermittent domain \emph{completely}, (ii) memory virtualization---reducing the execution speeds by \todo{provide numbers}{Przemek} of broad set of benchmarks executed on popular energy-harvesting embedded platform, and (iii) (adaptive) task coalescing---enabling fastest execution \todo{provide numbers}{Przemek} despite changing energy arrivals and enabling re-use of the same code on intermittent systems with varying energy reservoirs. \sys, with its task creation/execution automation, responses to the need that all existing predecessors of \sys (chronologically, Mementos, DINO, Chain, Alpaca) strongly advocated. We believe that \sys is a next major step in enabling computing community to enter the intermittent-power domain without big learning curve investments. Next logical step in \sys design is the introduction of features needed to implement first fully-operational operating system for transiently-powered devices.\todo{find a more convincing reason for a follow-up work}{Brandon} \todo{Revise this section}{Brandon}