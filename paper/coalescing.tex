\sys is able to virtually merge tasks to construct a bigger virtual task and commit the state of the virtual task instead of the individual real tasks. 

\subsubsection{Task Merging Algorithms}



\subsubsection{Power Interrupt Immune Scheduler}

% TNT :  Total Number of Tasks
% JT 	: Total Jump
% ID	: Task ID
% D	: relative Jump (Delta)
% VCT_PT : Current Task Pointer

% if(TJ < TNT)
% 	VCT_PT <- VCT_PT + D
% else
% 	while ((dis = TJ - TNT) > TNT)
% 		dis -= ID
% 	VCT_PT <- VCT_PT + dis


\begin{algorithm}[t]
	\caption{\sys's scheduler: relative jump algorithm}
	\label{algo:relativeJump}
	\scriptsize
	%\small
	\begin{algorithmic}[1]
			\State \textsf{TNT}: Total Number of Tasks
			\State \textsf{ID}: Task ID
			\State \textsf{$\delta$}: Relative Jump
			\State $\textsf{TJ} \leftarrow (\textsf{ID} + \delta )$ \Comment{Total Jump}
			\State \textsf{\textsf{$VCT_{pt}$}}: Virtual Current Task Pointer

			\If { \textsf{TJ} > \textsf{TNT} }
				\State $\textsf{dis} = \textsf{TJ} - \textsf{TNT}$
				\While{ $ \textsf{dis} > TNT $ }
					\State $\textsf{dis} -= \textsf{TNT}$
				\EndWhile
				\State $\textsf{dis} -= \textsf{ID}$
				\State \textsf{$VCT_{pt}$} $+= \textsf{dis}$
			\Else
				\State \textsf{$VCT_{pt}$} $+= \delta $

			\EndIf
	\end{algorithmic}
\end{algorithm} %Task jumping algorithm

It utilizes a persistent circular buffer (persistent linked list) to keep the state of a program across power failures. \sys provides an API to enable a programmer to have a full control over the execution flow of the program, i.e. (un)blocking a task or re-execute the same task which is particularly important in the intermittent execution to emulate a persistent loop. 

\paragraph{Fixed Virtual Task Size}

	\begin{algorithm}
		\caption{Task coalescing mechanism}
		\label{algo:fixVirtTask}
		\scriptsize
		%\small
		\begin{algorithmic}[1]
			\State $VT \subset \text{\{\sys Tasks\}} $  \Comment{$VT:$ Virtual Task}
			\State VTS : VT size
			\State MVTS: maximum VT size \Comment{line added in fixed size approach}
			\vspace{0.1cm}

			\While {$True$}
				\State $VT \leftarrow VT_{next}$
				\vspace{0.1cm}
				\While {execute $VT$} 
					\If { $\text{power failed twice}$ }				
							\State $VTS--$  
							\State $ MVTS = VTS $ \Comment{line added in fixed size approach}
						\EndIf
				\EndWhile

				\vspace{0.1cm}
				\If {$ \text{All tasks executed}$}
					\If{$VTS < MVTS$} \Comment{line added in fixed size approach}
					\State $VTS++$
					\EndIf
				\EndIf
			\EndWhile
		\end{algorithmic}
	\end{algorithm}

\begin{figure}
	\centering
	\includegraphics[width=0.8\columnwidth]{figures/virtualTaskSize.eps}
	\caption{Size of the virtual task versus the execution time of a dummy application that contains 12 empty tasks.}
	\label{fig:virtualTaskSize}
\end{figure}

\begin{figure}
	\centering
	%\includegraphics[width=0.25\columnwidth]{figures/task_coalescing}
	\caption{Task coalescing architecture.}
	\label{fig:}
\end{figure}

The question remains, is the best strategy to coalesce all tasks into a singe one, as the energy becomes fully available, i.e. from energy harvesting to continuous power environment. To answer this question we have implemented a simple program composed of $x$ empty tasks.  Figure~\ref{fig:virtualTaskSize} shows the benefit of virtualizing the Modular Intermittent Execution Mode (MIEM) in the best case scenario (continuous power supply). 


%\begin{algorithm}
%	\caption{Opportunistic virtual Task size}
%	\label{algo:fixVirtTask}
%	\scriptsize
%	%\small
%	\begin{algorithmic}[1]
%		\State $VT \subset \text{\{\sys Tasks\}} $  \Comment{$VT:$ Virtual Task}
%		\State VTS : VT size
%		\vspace{0.1cm}
%		
%		\While {$True$}
%		\State $VT \leftarrow VT_{next}$
%		\vspace{0.1cm}
%		\While {execute $VT$} 
%		\If { $\text{power failed twice}$ }				
%		\State $VTS--$  
%		\EndIf
%		\EndWhile
%		
%		\vspace{0.1cm}
%		\If {$ \text{All tasks executed}$}
%		\State $VTS++$
%		\EndIf
%		\EndWhile
%	\end{algorithmic}
%\end{algorithm}