\sys is a new programming and execution model for intermittent computing on energy-harvesting devices. \sys addresses the challenges outlined in Section~\ref{sec:background} to make task-based intermittent programs {\em programmable} and {\em efficient}. \sys accomplishes this goal with a constellation of a new programming model and run time software system support, that supports dynamically adaptive task-based execution. Figure~\ref{fig:system_overview} shows an overview of \sys.

\textbf{\sys Programming and Execution Model.}  To use \sys, a programmer first writes plain, imperative C code. The programmer then must decompose the program into tasks. To manually decompose a program into tasks, the programmer designates a set of functions as tasks, sequences control-flow between these tasks, and annotates memory accesses that manipulate data shared by multiple tasks. In other words, it requires reasoning similar to prior task-based systems~\cite{chain,alpaca}. 

\begin{wrapfigure}{t!}{0.5\textwidth}
	\centering
	\includegraphics[width=0.5\columnwidth]{figures/graffle/overview.pdf}
	\caption{\sys top-level view. \sys is composed of two core components: \emph{Task Scheduler} and \emph{Memory Manager}.\todo{Redraw figure to map new terms}{Carlo}}
	\label{fig:system_overview}
\end{wrapfigure}

A programmer may also opt to use compiler support to automatically decompose a program into tasks, leveraging recent work~\cite{cleancut_2018,baghsorkhi_cgo_2018}. Without loss of generality, we assume throughout this work that the programmer manually decomposed the program into tasks; \sys's behavior with automatically decomposed code would be identical.

The programmer compiles their task-based code, and links to the \sys runtime,
producing a \sys-enabled binary. The \sys runtime implements \sys's task-based
programming and execution model. The runtime system also includes \sys's novel
{\em virtualizing memory manager} and its {\em adaptive task scheduler}, both
of which are essential to \sys's adaptive task-based execution.

\textbf{\sys Adaptive Task Scheduler.} 
\sys's adaptive task scheduler decides when to coalesce consecutive tasks
together to amortize commit and transition costs, and when to downscale a task,
breaking it into multiple tasks, to avoid non-termination.

\sys's scheduler dynamically {\em coalesces} statically\hyp{}defined tasks to
avoid inessential overheads associated with completing tasks. By default, tasks
run in a sequence and each task commits its task-shared state as it completes.
The time and energy cost of a task's commit is unnecessary if the task and its
successor both complete without a power failure. The key insight that \sys
leverages is that the first task could have deferred its commit to the second
task. {\em Coalescing} tasks by deferring the first commit avoids the fixed
cost of the first commit (task transitioning overheads). Coalescing also
amortizes per-variable commit cost, committing each location accessed by both
tasks only after the second task, rather than once per task.  \sys makes
\emph{energy-aware} scheduling decisions, using a short history of recently
executed tasks to estimate available energy and guide future decisions. \sys
relies on {\em software only}, avoiding the need for additional circuitry to,
e.g., measure available energy.

%to set the adequate coalesced task
%size accordingly. This fundamental feature prevents \sys from blind coalesced
%task adaptation which may result in a repeated power failure problem. For
%example, If \sys enlarges the coalesced task size by $x$ static tasks on a
%coalesced task completion and reduces it by the same number of static tasks on
%a power reboot, then \sys may need to reboot, for instance, ten times to reduce
%a coalesced task size from ten to one, when $x=1$. Details of coalescing will
%be given in Section~\ref{sec:task_coalescing}.

\sys's scheduler {\em downscales} a task to preserve forward progress when a
single task repeatedly fails to execute to completion.  For a repeatedly-failing
task, the scheduler uses a timer that fires at a decreasing interval to
determine where to execute a {\em partial commit} of the task. Partial
commit saves progress through a task that runs for too long to complete
using the device's buffered energy. Eventually, after a series of partial
commits, the too-long task completes --- partial commit favors progress over task atomicity.  
When a task repeatedly fails, the scheduler uses a timer that counts an operating interval
that decreases in length with each failure and \sys partially commits at the end of the interval. 
Section~\ref{sec:task_coalescing} provides details of \sys's task scheduler.
%and relies on the virtual memory manager to gain a full \emph{flexibility} on
%where to split a task. These features distinguish \sys task splitting technique
%from the traditional checkpointing. A checkpointing system that uses voltage
%threshold to place a checkpoint might cause a program to crash if the
%checkpoint happened to be placed between a peripheral initialization and its
%call. If a programmer disables interrupts between the peripheral initialization
%and the call, the program may fail indefinitely to progress since the
%checkpoint may only be deferred. The \sys task downscaling manager, on the
%other hand, triggers a \emph{partial commit} faster than the previous one on a
%repeated failure to guarantee forward progress. Details of task downscaling
%will be given in Section~\ref{sec:task_downsizing}.

\textbf{\sys Virtual Memory Manager.} \sys is able to efficiently coalesce
tasks because of its efficient virtual memory manager. \sys's memory manager
paginates memory and ensures that data in a page remain consistent despite
power interruptions. \sys allows a task to manipulate data in a volatile copy
of a page only. Pages swap between volatile and non-volatile memory, depending
on the capacity of the volatile memory and the program's access pattern. \sys
tracks a task's memory accesses efficiently at page granularity (rather than
using, e.g., costly word-granular tracking). When a task ends, each page it
accessed commits from volatile memory (or from a non-volatile swap region for
dirty pages) back to the non-volatile main memory. Pages efficiently,
atomically commit using a two-phase commit procedure accelerated using hardware
support for direct memory access (DMA). Section~\ref{sec:memory_virtualization}
provides details of \sys's memory manager.
