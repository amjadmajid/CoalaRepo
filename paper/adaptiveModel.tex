
%% Coalescing motivation
Finding an optimal task size, given random energy conditions is an open question. Obviously, a static approach does not have the potential to answer it. Therefore, \sys champions runtime-based methods to approximate the ideal task size under random energy shots arrival. Furthermore, \sys advocates making intermittently powered software energy-aware is the key for \emph{efficient code execution and probability}. 
%However, given the extremely limited recourses any optimization technique must adhere to the principle of simplicity, otherwise it introduces a non-tolerable overhead. 

% %% address the challenge and define the optimal task
\sys advances its execution in a state-less (or virtual) manner, and then it frequently saves its forward progress. The longer \sys virtually progresses, the less committing (saving data to non-volatile memory) overhead it introduces. However, long state-less execution results in a considerable re-execution penalty---all the tasks that have been virtually executed must be re-executed after a power interrupt. As such, an optimal task is a task that occupies with a single commit an entire power cycle, which is therefore necessarily of a varying length.  

