\noindent \textbf{Intermittently-Powered Devices:} There is plentiful of research results available on energy harvesting, or more specifically wireless power provision, for embedded devices like sensors or microcontrollers. This paragraph is too short to cover this area extensively and we refer the reader to overview papers~\cite{prasad_comst_2014,sample_procieee_2013,huang:commag:2015,visser_procieee_2013,kamalinejad_commag_2015,ku_cst_2016} for in-depth discussion. With power consumption of electronics going down dramatically with every year, computing systems can be now powered completely from ambient energy. This opens new research areas such as radio frequency-powered computers~\cite{patel_pervasive_2017,rf_powered_computing_gollakota_2014}. Indeed, plenty of new embedded platforms appear that are powered solely from ambient (intentional or non-intentional) energy and include~\cite{wisp5,moo,zhao_rfid_2015,holleman_biocas_2008,thomas_jbcs_2012,naderiparizi_rfid_2015,rodriguez_tbcs_2015,liu_sigcomm_2013,kicksat,nadeau_naturebio_2017}. But there is a price tag associated to them: harvested energy is miniscule, unstable, difficult to predict ergo \emph{intermittent}. Therefore, to guarantee completion of computing, programming and runtime models (and measurement and support systems such as~\cite{hester_sensys_2014,hester_sensys_2015,edb} which are beyond the scope of this paper) are being actively researched that are specifically tailored to these intermittently-powered systems. \todo{Revise this section}{Brandon}

\noindent \textbf{Intermittent Execution Environments:} Earlier works on runtime environments for energy-harvesting (volatile memory-only) systems include Dewdrop~\cite{dewdrop} and Eon~\cite{sorber_sensys_2007}, in which it is ensured (by a programmer or a scheduler) that task will fit the available remaining energy. No intermittency and data consistency issues are considered. Similar, specific-application tailored system includes Wisent and Stork~\cite{stork,wisent}. DINO~\cite{dino} versions non-volatile memory and checkpoints volatile program state which is memory and energy demanding. Rachet~\cite{ratchet}, implementing checkpointing as well, assumes complete non-volatile hardware. Clang~\cite{hicks_isca_2017} assumes a dedicated hardware for idempotency violation detection. On-demand checkpoint runtimes include Mementos~\cite{mementos}, Quickrecall~\cite{quickrecall}, Hibernus/Hibernus++~\cite{hibernus,hibernusplusplus} and HarvOS~\cite{mottola2017harvos}. They all require hardware support to measure remaining storage energy---a feature prohibitive given typical high energy cost expenditure of such hardware. The only available task-based intermittent runtime systems, free from hardware assumptions or heavy checkpointing, are Chain~\cite{chain} and Alpaca~\cite{alpaca}. \sys goes beyond what any task-based system performed by providing \emph{automatic} task creation and \emph{adaptivity} to energy state and energy storage \emph{obliviousness} of the compiled code. \todo{Revise this section}{Brandon}

\noindent \textbf{Memory Virtualization:} Review of memory virtualization/transactional memory environments (not necessarily for intermittent power devices). \todo{Fill-in this section}{Alexei}