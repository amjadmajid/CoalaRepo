\noindent \textbf{Intermittently-Powered Devices:} There is plentiful of research results available on energy harvesting, or more specifically wireless power provision, for embedded devices like sensors or microcontrollers. This paragraph is too short to cover this area extensively and we refer the reader to overview papers~\cite{prasad_comst_2014,sample_procieee_2013,huang:commag:2015,visser_procieee_2013,kamalinejad_commag_2015,ku_cst_2016} for in-depth discussion. With power consumption of electronics going down dramatically with every year, computing systems can be now powered completely from ambient energy. This opens new research areas such as radio frequency-powered computers~\cite{patel_pervasive_2017,rf_powered_computing_gollakota_2014}. Indeed, plenty of new embedded platforms appear that are powered solely from ambient (intentional or non-intentional) energy and include~\cite{wisp5,moo,zhao_rfid_2015,holleman_biocas_2008,thomas_jbcs_2012,naderiparizi_rfid_2015,rodriguez_tbcs_2015,liu_sigcomm_2013,kicksat,nadeau_naturebio_2017}. But there is a price tag associated to them: harvested energy is miniscule, unstable, difficult to predict ergo \emph{intermittent}. Therefore, to guarantee completion of computing, programming and runtime models are being actively researched that are specifically tailored to these intermittently-powered systems. \todo{Revise this section}{Brandon}

\noindent \textbf{Intermittent Execution Environments:} Review of intermittent runtime environments. \todo{Fill-in this section}{Przemek}

\noindent \textbf{Memory Virtualization:} Review of memory virtualization environments (not necessarily for intermittent power devices). \todo{Fill-in this section}{Przemek}