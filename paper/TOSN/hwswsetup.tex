We prototype \sys and use its API to implement a set of benchmark applications representative of the embedded domain. We build the applications and deploy the binaries onto a real energy-harvesting device. 

%In this section we describe the setup for the experiments presented in Section~\ref{sec:evaluation} and characterize the benchmark applications.

\subsection{Experimental Setup}
\label{sec:results_hardware_software}
%
We used three different setups to evaluate \sys: 
(i) an RF-powered energy-harvesting device, WISP\,5.1~\cite{wisp5,wisp}, with a fixed capacitor size of 47\,$\mu$F; 
(ii) an MSP-EXP430FR5969 launchpad~\cite{MSP-EXP430FR5969_launchpad} powered by a BQ25570~\cite{BQ25570EVM-206_website} solar power harvester that is connected to an IXYS SLMD121H04L solar cell~\cite{SLMD121H04L_website} for experimenting with different energy buffers; and
(iii) an MSP-EXP430FR5969 launchpad~\cite{MSP-EXP430FR5969_launchpad} that is continuously powered.

Every benchmark used in Section~\ref{sec:evaluation} was run repeatedly on
each platform for a few minutes to ensure capturing a diverse power 
trace. The exact number of iterations depends on the ambient power intensity and the application itself.
In our experiments, the number of complete runs ranges from 4 to 125.


WISP contains the MSP430FR5969~\cite{wolverine} MCU with 64\,kB of
non-volatile (FRAM) memory and 2\,kB of volatile (SRAM) memory and was
configured to 1\,MHz clock speed.
We powered WISP using an RF signal generator emitting a 20\,dBm sinusoidal wave at 915\,MHz.
The signal generator was connected to the Laird RFMAX S9028PCRJ 8\,dBic
antenna~\cite{atlas2015}.
The antenna was oriented towards and in parallel with WISP's antenna, and
no objects obstructed the path.
We affixed WISP with a paper harness at the edge of a table at a height, from the table surface, of 10\,cm.
For distance-controlled experiments we positioned WISP at $d=\{15, 30,
50\}$\,cm from the exciter antenna.
To obtain execution time, the software toggled GPIO pins at sections of the code
under profile, and the Saleae~\cite{saleae} logic analyzer measured
intervals between edges in the signal. For continuous power experiments, the 
execution time was measured using the clock features in TI Code Composer Studio IDE version 7.1.
%
\subsection{Software Benchmarks}
\label{sec:software_benchmarks}

\begin{table}%
\caption{Characteristics of benchmarks used for evaluation.}
\label{tab:one}
\begin{minipage}{\columnwidth}
\begin{center}
\begin{tabular}{llll}
  \toprule
		App.&Tasks&SLOC&Description\\
		\hline
        \emph{ar} &10 &428 & activity recognition using a KNN\\ % with a deterministic input from a randomly-generated sequence of accelerometer data\\
        \emph{bc} &10 &371 & several bitcount algorithms\\
        \emph{cuckoo} &14 &426 & Cuckoo Filter with pseudo-random values\\
        \emph{dijkstra} &5 &198 & Dijkstra shortest path algorithm \\
        \emph{fft} &8 &449 & Fast Fourier Transform\\ % on three pre-generated 128 sample vectors\\
		\emph{sort} &4 &167 & selection sort algorithm\\
  \bottomrule
\end{tabular}
\end{center}
\end{minipage}
% \caption{Characteristics of benchmarks used for evaluation.}
\label{table:benchmark_table}
\end{table}%

We evaluated \sys using six benchmarks that are often used in embedded systems (summarized in Table~\ref{table:benchmark_table}). All applications were compiled using MSP430 GCC~\cite{ti-gcc} version 6.4.0 with \texttt{-O1} as optimization flag. The source code for all benchmarks will be released at \cite{coala_website}.

\paragraph{Comparison with Alpaca using the GCC compiler.}

We compare \sys against Alpaca~\cite{alpaca}, the state-of-the-art task-based system for intermittent computing. For a fair comparison, the task decomposition of the benchmarks is ensured to be the same for both systems.
%
Since Alpaca's compiler pass is implemented only for LLVM, we could not use that implementation to instrument the benchmarks and compile them with GCC. Instead, we \emph{manually} performed the instrumentation done by the compiler pass. The instrumentation consists of identifying WAR dependencies  and adding code and memory allocations to make a private copy of the affected variables.
By compiling both systems with the same compiler (GCC), we ensure that our comparison in Section~\ref{sec:evaluation} is fair.
