\paragraph{Intermittently-powered Devices.}
There is a large body of research on energy-harvesting and ambient-powered embedded devices summarized by~\cite{prasad_comst_2014,sample_procieee_2013,huang:commag:2015,visser_procieee_2013,kamalinejad_commag_2015,ku_cst_2016}. For instance, a new wave of embedded systems powered by radio waves is emerging~\cite{patel_pervasive_2017,rf_powered_computing_gollakota_2014,wisp5,moo,zhao_rfid_2015,holleman_biocas_2008,thomas_jbcs_2012,naderiparizi_rfid_2015,rodriguez_tbcs_2015,liu_sigcomm_2013,kicksat,nadeau_naturebio_2017}. The emergence of such hardware platforms
has led to the development of instrumentation, debugging, and prototyping tools for such systems
~\cite{hester_sensys_2014,hester_sensys_2015,edb,capybara,stork,tan_infocom_2016,flicker}.

\paragraph{Checkpointing-based Systems.}
Early work on energy-harvesting runtimes, such as Dewdrop~\cite{dewdrop}, assumed simple computations that complete on a predictable burst of energy.
%
%Eon~\cite{sorber_sensys_2007} has scheduled such computation based on priority.
%
Support for computation that spans power failures was first achieved with statically-placed conditional checkpoints in Mementos~\cite{mementos}. 
%Mementos left the possibility of non-volatile state becoming inconsistent with volatile state saved in the checkpoint, rendering in-place writes to data structures in non-volatile memory unsafe~\cite{mspcdino}. 
DINO~\cite{dino} addressed the consistency problem by selectively versioning non-volatile state within the checkpoints. Ratchet~\cite{ratchet} ensured consistency by placing a checkpoint at the beginning of each idempotent region in the code. Ratchet has a similar technique to task splitting with a core difference: the checkpoint is of a fixed size, while \sys privatizes a varying number of memory pages.
%
%Ratchet has a similar technique to task splitting with some essential differences: (i) It does not need to handle non-volatile memory accesses, since it assumes a unified non-volatile memory, whereas \sys has to save the volatile pages in non-volatile memory; (ii) The checkpoint is of a fixed size, while \sys privatize variant number of memory pages. 
%
Clank~\cite{hicks_isca_2017} ensured consistency with custom hardware that dynamically tracks WAR dependencies in memory accesses and checkpoints on demand. The consistency problem was also approached with a combination of undo- and redo-logging in software~\cite{baghsorkhi_cgo_2018}. Just-in-time checkpointing, such as Quickrecall~\cite{quickrecall} and Hibernus++~\cite{hibernusplusplus}, eschews inconsistency by saving all volatile state immediately before a power failure and halting the execution. Unlike \sys, such systems rely on introspection hardware to monitor supply voltage and on accurate worst-case bounds on checkpoint cost.

In all of the above systems, with the exception of DINO~\cite{dino}, checkpoints are dynamic, i.e., the programmer does not have explicit control over the point at which the code may be resumed after power failure. Dynamic checkpointing systems make it difficult for the programmer to respect application-level atomicity constraints, such as correlating sensor readings. Checkpointing systems that copy most volatile state scale poorly as the size of volatile memory increases.  QuickRecall, Clank, and Ratchet reduce the copying overhead by allocating the stack in non-volatile memory, whichrequires more time and energy to access than volatile memory (cf. Section~\ref{sec:cost_task-based}) and not a viable option for off-chip non-volatile memory. In addition, QuickRecall and Clank require custom hardware. 

\paragraph{Task-based Systems.}
Alternatives to checkpointing are recent systems based on static tasks, such as Chain~\cite{chain}, Alpaca~\cite{alpaca}, inK~\cite{yildirim2018ink}, and Mayfly~\cite{hester_sensys_2017}. Using static tasks, they eliminate the need to checkpoint volatile state. Using channel-based memory models~\cite{chain,hester_sensys_2017} or automatic privatization and redo-logging~\cite{alpaca} they avoid checkpointing overheads. Moreover, task-based models facilitate respecting application-level atomicity constraints. \sys also relies on
statically defined tasks to avoid checkpointing volatile state. However, unlike prior systems, \sys coalesces its statically-defined tasks at runtime into more efficient dynamic tasks that adapt to changing energy conditions. \sys's mechanism for ensuring memory consistency also differs from the channel-based~\cite{chain} and privatization-based~\cite{alpaca} mechanisms in prior systems. \sys keeps memory consistent through memory virtualization optimized for bulk accesses to task-shared data with high locality.

\paragraph{Task Decomposition.}
In contrast to \sys's construction of coalesced tasks at run-time, prior work has proposed to optimize task size at compile-time. CleanCut~\cite{cleancut_2018} program analysis statically estimates energy consumption and splits the program into tasks until all tasks consume less energy than the device can store. An alternative program analysis generates different versions of a program with different task sizes and empirically selects the best among them~\cite{baghsorkhi_cgo_2018}. HarvOS~\cite{mottola2017harvos} takes a hybrid approach that uses a program analysis to place a minimal number of conditional checkpoints that test the energy level at run-time before copying state, like Mementos~\cite{mementos}. Unlike \sys, such compiler-based approaches face the challenge of statically predicting energy consumption of arbitrary input-dependent code with peripheral access, which is a problem without a general solution. Furthermore, a static decomposition approach prevents portability across devices with different storage capacitors. In contrast, \sys avoids forcing any assumptions at compile time and adapts to energy storage capacity and incoming energy conditions at run-time.

\paragraph{Memory Virtualization.}
Prior work on embedded systems has studied a variety of memory virtualization strategies relating to \sys.
TinyOS~\cite{levis2005tinyos} and nesC~\cite{nesc} support dynamic memory management. Later work extended the memory manager to support memory virtualization backed by flash memory~\cite{sensornetvm} and to ensure memory and type safety~\cite{tinyosmemorysafety}. SOS~\cite{sos}, Contiki~\cite{contiki}, and t-kernel~\cite{tkernel} also developed memory management abstractions that virtualize memory size and provide safe and indirect access. Mat\'e~\cite{mate} developed full virtual machine support for sensor nodes, virtualizing not just memory resources, but other states and peripherals. The goal of \sys is to provide consistent, intermittent execution, leveraging the benefits of efficient bulk copying. In contrast, prior efforts focused more on programmability and run-time reliability properties provided by virtual memory.

A related domain is unbounded, page-based transactional memory and deterministic parallel runtime systems~\cite{pagebasedtm,grace}.  These works have a different mechanism and purpose than \sys---ensuring that data are consistent and deterministically updated during concurrent executions. Their similarity with \sys is in managing state to ensure consistency at the granularity of pages to amortize checking and tracking costs. \sys's paging implementation, which keeps a shadow page for each page to use during commit, is similar to the shadow paging scheme used for transactional commit~\cite{pagebasedtm}.
