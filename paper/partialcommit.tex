\subsection{Task Downsizing}
\label{sec:task_downsizing}

We proceed with the second task adaptation feature of \sys, namely reduction of the task size boundary to enable execution of a program on any energy buffer size.

\begin{figure}
	\caption{\todo{Figure to describe partial commit is needed}{Sinan, Carlo}}
\end{figure}

\subsubsection{When Task Downsizing is Triggered}

If power fails consecutively twice before the system can complete one single static task, it means that the static task requires more energy than currently available in a discharge cycle, so it is necessary to have a mechanism to execute a chunk of the task, save the partial progress through the task, and continue try to run the rest of the task. If a power failure occurs after saving partial task progress, the system makes forward progress, albeit without task atomicity. \sys may save partial task progress multiple times before completing a very long task. To avoid the fail-stop case of non-termination, \sys's partial commit mechanism will split tasks that have not been marked as atomic by the programmer.

A partial commit saves the stack, registers and the volatile pages to the non-volatile memory. \sys starts with a big random guess and down-scales it using a binary search until it succeeds in preserving the partial execution progress. The obtained value is the initial seed for the next partial commit procedure.

\textcolor{red}{More variables describing the partial commit algorithm: we do not explore as many options for partial commit as we do for coalescing}

\todo{Proof of the maximum granularity of task downscaling - show it is limited by the clock}{Carlo}

\subsubsection{Why Task Downsizing is not \sys's Default Behavior}

Task downsizing is the core novelty of \sys when it comes to code portability and re-usability. However, it has limitations, which force  it to be a non-default \sys behavior. 

\textbf{Static deployment of partial commit stubs.} This approach reduces programmer effort, although the programmer still needs to apply \sys's API and distribute trigger points for partial commits. On the other hand, \sys' downsizing introduces an additional overhead by saving the stack and the register file each time it commits the pages. 

\textbf{Dynamic deployment of partial commit stubs.} \sys downsizing does not rely on any information from the programmer to trigger the commit process. However, \sys utilizes a trial and error approach to optimize the commit rate over the entire program execution time. Therefore, it suffers from a large cumulative re-execution penalty.