% \begin{wrapfigure}{t!}{0.45\textwidth}
\begin{figure}
	\centering
	\includegraphics[width=\columnwidth]{figures/overview.pdf}
	\caption{System's top-level view. \sys is composed of two core components: \emph{Adaptive Task Scheduler} and \emph{Virtual Memory Manager}.}
	\label{fig:system_overview}
% \end{wrapfigure}
\end{figure}

\sys is a new programming and execution model for intermittent computing on energy-harvesting devices. \sys addresses the challenges outlined in Section~\ref{sec:background} to make task-based intermittent programs {\em programmable} and {\em efficient}. \sys accomplishes this goal with a constellation of a new programming model and runtime software system support, that supports dynamically adaptive task-based execution. Figure~\ref{fig:system_overview} shows an overview of \sys.

\textbf{Programming and Execution Model.}  To use \sys, a programmer must convert a plain C code into tasks by encapsulating the code in a top-level set of functions, sequences the control-flow between these tasks, and annotates memory accesses that manipulate global data.
% In other words, it requires reasoning similar to prior task-based systems~\cite{chain,alpaca}. 

% A programmer may also opt to use compiler support to automatically decompose a program into tasks, leveraging recent work~\cite{cleancut_2018,baghsorkhi_cgo_2018}. Without loss of generality, we assume throughout this work that the programmer manually decomposed the program into tasks; \sys's behavior with the automatically decomposed code would be identical.

The programmer compiles their task-based code, and links to \sys runtime,
producing a \sys-enabled binary. The runtime library relies on \sys's novel
{\em virtual memory manager} and its {\em adaptive task scheduler} to adapt its execution with the energy conditions.

\textbf{Adaptive Task Scheduler.} 
\sys's adaptive task scheduler makes \emph{energy-aware} scheduling decisions to group tasks together to amortize commit and transition costs, or to downscale a task, breaking it into multiple tasks, to avoid non-termination. \sys's scheduler uses its recent execution history---no hardware dependency---as a metric to estimate energy availability. 


\sys's scheduler {\em downscales} a task to preserve forward progress when a
single task repeatedly fails to execute to completion.  For a repeatedly-failing
task, the scheduler uses a timer that fires at a decreasing interval to
determine where to execute a {\em partial commit} of the task. Partial
commit saves progress through a task that runs for too long to complete
using the device's buffered energy. Eventually, after a series of partial
commits, the too-long task completes---partial commit favors progress over task atomicity.  
When a task repeatedly fails, the scheduler uses a timer that counts an operating interval
that decreases in length with each failure and \sys partially commits at the end of the interval. 
Section~\ref{sec:task_coalescing} provides details of \sys's task scheduler.
%and relies on the virtual memory manager to gain a full \emph{flexibility} on
%where to split a task. These features distinguish \sys task splitting technique
%from the traditional checkpointing. A checkpointing system that uses voltage
%threshold to place a checkpoint might cause a program to crash if the
%checkpoint happened to be placed between a peripheral initialization and its
%call. If a programmer disables interrupts between the peripheral initialization
%and the call, the program may fail indefinitely to progress since the
%checkpoint may only be deferred. The \sys task downscaling manager, on the
%other hand, triggers a \emph{partial commit} faster than the previous one on a
%repeated failure to guarantee forward progress. Details of task downscaling
%will be given in Section~\ref{sec:task_downsizing}.

\textbf{Virtual Memory Manager.} \sys is able to efficiently coalesce
tasks because of its efficient virtual memory manager. \sys's memory manager
paginates memory and ensures that data in a page remain consistent despite
power interruptions. \sys allows a task to manipulate data in a volatile copy
of a page only. Pages swap between volatile and non-volatile memory, depending
on the capacity of the volatile memory and the program's access pattern. \sys
tracks a task's memory accesses efficiently at page granularity (rather than
using, e.g., costly word-granular tracking). When a task ends, each page it
accessed commits from volatile memory (or from a non-volatile swap region for
dirty pages) back to the non-volatile main memory. Pages efficiently,
atomically commit using a two-phase commit procedure accelerated using hardware
support for direct memory access (DMA). Section~\ref{sec:memory_virtulaization}
provides details of \sys's memory manager.
