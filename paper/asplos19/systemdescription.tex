\begin{figure}
	\centering
	\includegraphics[width=\columnwidth]{figures/system-overview.pdf}
	\caption{\sys top-level view. APP: application, ATS: adaptive task scheduler, VMM: Virtual Memory Manager, PM: physical memory.}
	\label{fig:system_overview}
\end{figure}

\sys is a new programming and execution model that supports adaptive task-based execution and eliminates restructuring and re-compilation of applications considering device's energy buffer. \sys addresses the challenges given in Section~\ref{sec:background} by making task-based intermittent applications portable in the sense of different energy storage sizes, while keeping their execution efficient. Fig.~\ref{fig:system_overview} shows an overview of \sys.

\paragraph{Programming and Execution Model.}
To use \sys, a programmer must (i) convert a plain C code into tasks by encapsulating the code in a top-level set of functions, (ii) sequence the control-flow between these tasks, and (iii) annotate memory accesses that manipulate task-shared data. Then, they compile and link the code against \sys's runtime, producing a \sys-enabled binary. The runtime relies on \sys's novel {\em adaptive task scheduler} to adapt its execution with the energy conditions. Facilitating efficient task adaptation requires dynamic memory protection which \sys's \emph{virtual memory manager} handles through \emph{page privatization}.

\paragraph{Adaptive Task Scheduler.}
\sys's adaptive task scheduler (ATS) makes \emph{energy-aware} scheduling decisions to group tasks together or split a task. By coalescing tasks \sys \emph{amortizes commit and transition costs}, and by splitting a task, after it repeatedly failed to complete, it \emph{avoids the task non-termination problem}. The scheduler uses its recent execution history---i.e., it is hardware independent---as a metric to estimate energy availability and eventually to decide on the coalesced task size. Section~\ref{sec:task_adaptation} describes ATS.

\paragraph{Virtual Memory Manager.}
\sys virtual memory manager (VMM) is the key enabler to ensure memory consistency while coalescing tasks. VMM allows applications to interface with only fast volatile memory pages and privatizes the pages demanded by a coalesced task in order to solve a novel data consistency problem; namely \emph{task coalescing-induced WAR dependency}. VMM achieves page privatization by keeping all page modifications in non-volatile memory on a coalesced task transition. Section~\ref{sec:memory_virtulaization} outlines VMM.
