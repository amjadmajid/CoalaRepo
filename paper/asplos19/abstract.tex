Freeing embedded devices from batteries is a key factor in developing the energy neutral Internet of Things.
Energy harvesting, however, is unpredictable and can only power a device intermittently.
Therefore, the two paradigms of intermittent execution (checkpointing and task-based) save the intermediate program state into non-volatile memory frequently, to preserve the execution progress of a program.
Despite the superiority of the task-based approach, tasks' size is fixed at compile-time, and agnostic to energy conditions.
% it fixes the task size at compile time and operates obliviously to changing energy conditions.
% When tasks are sized without accounting for available energy,
Thus,
the state may be persisted either more frequently than necessary or not frequently enough for the program to make progress and terminate.
To address these challenges, we propose \sys, an \emph{adaptive} and \emph{efficient} task-based execution model.
\sys progresses on a multi-task scale when energy permits and preserves the computation progress on a sub-task scale if necessary.
\sys ensures that power failures do not leave the program state in non-volatile memory inconsistent using a specialized memory virtualization mechanism.
Our evaluation on a real energy-harvesting platform shows that \sys reduces run time by up to 54\%, and by 26\% on average, as compared to a state-of-the-art system, and is able to progress where static systems fail.
% Our evaluation shows that \sys is $2X$ faster than the state-of-the-art system in most cases and is able to progress where static systems fail.
