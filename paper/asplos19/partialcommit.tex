\subsection{Task Downscaling}
\label{sec:task_downsizing}

\sys uses {\em task downscaling} to make progress through a task that is too
large to complete using the energy that the device can buffer.  Task
downscaling executes part of the long task and interrupts its execution with a
{\em partial commit}, after which the long task continues. If power fails, the
partial commit preserves the progress through the task, and execution resumes
from the point of the partial commit, rather than from the start of the task.
Eventually, after some number of partial commits and power failures, the task
will complete and execution will proceed to a task with an energy demand
accommodated by the device's energy buffer. Task downscaling violates task
atomicity in a non-terminating execution, favoring continued progress over
atomicity. If a program requires a task's atomicity over its progress, 
the programmer can disable task downscaling for that task.

The key design issue for task downscaling is deciding when to partially commit.
\sys only applies task downscaling to a task that is likely to be
non-terminating.  A task is likely to be non-terminating if it fails to execute
completely twice consecutively. The second incomplete execution executes after
a power failure, when the device will have fully recharged its energy buffer.
If a task cannot complete when executing with a full energy buffer, the task is
non-terminating.

After identifying a task as likely non-terminating, \sys must decide when
during the task's execution to partially commit the task's state.
%
\sys sets a timer at the start of the likely non-terminating task, initializing
it to a very large value (e.g., an estimate of the maximum execution time using
the device's energy buffer).
%
If the timer expires before power fails, \sys partially commits, retaining 
the timer value to use for future partial commits. 
%
If the timer does not expire before power fails, \sys halves the timer and
tries again.
%
The exponential decrease of the timer value converges rapidly to a usable one.
