Advances in processor efficiency along with the development of energy-harvesting power systems has created a new category of devices that require neither a battery nor a tethered power supply. These devices operate entirely using energy extracted from their environment, such as RF energy~\cite{}, photovoltaics~\cite{}, and vibration~\cite{}. Incorporating compute, storage, sensing, and communication hardware, such energy-independent devices are a promising candidate as the underlying technology for emerging Internet of Things (IoT) applications, in- and on-body medical applications~\cite{}, and emerging applications, like energy-harvesting nano-satellites~\cite{}.

While promising, energy-harvesting devices present a unique design challenge in
that they operate only {\em intermittently} when energy is available. An
energy-harvesting device buffers energy from its environment, e.g., in a
capacitor. When a threshold quantity of energy accumulates, the device begins
operating. Harvestable energy sources typically produce relatively very little
power compared to a platform's operating power level. A devices operates for a
brief period of time and when buffered energy is depleted, shuts down and
begins recharging to later operate again. Charge and discharge times vary by
device and some devices~\cite{wisp} may fail 10 to 100 times per second~\cite{}.

Software in an energy-harvesting system operates according to the {\em
intermittent execution model}~\cite{dino}, with bursts of execution that are
interrupted by failure periods. When power fails, a device loses its volatile
state, i.e., registers, stack, SRAM, and retains its non-volatile state, i.e.,
FRAM, Flash. While capturing periodic checkpoints~\cite{mementos,tictpl,quickrecall}
and sleep scheduling~\cite{dewdrop,hibernus,hibernusplusplus} help to preserve
execution progress, failures can leave volatile and non-volatile state inconsistent, leading to unrecoverable failures~\cite{mspcdino,edb,edbtoppicks}. 

\TODO{Don't forget to include all the references in this section!}

%FRAM costs more than FRAM -- need to figure out how to execute out of SRAM
New programming and execution models~\cite{dino,ratchet,chain,alpaca} and
architecture mechanisms~\cite{clank,idetic,nvp} help avoid data
inconsistency. New architectures require costly hardware changes and are
inapplicable to today's systems. New programming models and compilers force
programmers to use new memory models~\cite{chain,ratchet}, which may inhibit
adoption. New memory models may also increase the amount of non-volatile
memory accesses by requiring non-volatile versioning~\cite{chain} or completely
precluding the use of volatile memory~\cite{ratchet}. Increasing pressure on
non-volatile memory or precluding the use of volatile memory may degrade
efficiency; e.g., FRAM has a higher access energy and latency than
SRAM~\cite{alpaca,wisp}. Commercially available devices are often provisioned
with much less SRAM than FRAM~\cite{wolverine,wisp} and effectively leveraging
the relatively high efficiency of SRAM is a key challenge for task-based
intermittent systems.

\TODO{concrete cost multiples here?}

%Task sizing is hard -- how to chop without risking inefficiency or non-term
Several recent programming efforts~\cite{alpaca,chain} proposed {\em
task-based} programming models, that require the programmer to statically
decompose their program into tasks. A task is like a function with no callers
that can include arbitrary computation and, upon completion, is guaranteed to
have executed {\em atomically} and to have logically {\em committed} memory
updates. The programmer expresses the flow of control from one task to another
and task control-flow may be input-dependent. An important programming
challenge in task-based models is that the length of execution of a task is
limited by the total amount of energy that a device can buffer. Assuming that
input power is negligible compared to operating power, a task will never be
able to complete if its execution consumes more energy than the system can
buffer. Transitioning from one task to the next imposes a cost and excessively
small tasks may be inefficient. Appropriately sizing tasks to ensure progress
and high performance is a key challenge for task-based intermittent systems. 

In this work, we develop \sys: a new programming and execution model that uses
{\em volatile memory virtualization} to leverage efficient volatile memory and
and {\em dynamic task coalescing} to efficiently execute tasks without
exceeding available energy. \sys's memory virtualization mechanism uses a
device's SRAM as the working memory during a task's execution, populating the
SRAM with a page of data from FRAM as necessary at the start of the task. At
the end of the task, the SRAM is committed back to FRAM using efficient DMA
block copies. \TODO{Brandon sez: Is the next sentence what the system does?
Or are we doing paging?} When a task accesses data outside of a page in SRAM,
the task must fetch data from FRAM. We explore two strategies. The first
strategy is {\em demand paging}, which swaps the SRAM page with a new page from
FRAM, buffering the swapped-out page until commit. The second strategy is {\em
buffered direct access} \TODO{better name}, which directly reads and writes
FRAM relying on dynamic double-buffering to ensure memory is consistent at
commit.

\sys's dynamic task coalescing mechanism allows the programmer or a compiler to
intuitively decompose their program into small tasks that amortize fixed
per-task overheads, yet present no risk of exceeding device energy capacity. As
such a decomposition executes, \sys {\em coalesces} dynamically consecutive
tasks. Coalescing elides the commits of coalesced tasks by buffering multiple
tasks' updates in an FRAM commit buffer. Periodically, as the span of the
coalesced task grows, \sys ends coalescing and commits the state of the
coalesced task. If a power failure interrupts a sequence of coalesced tasks,
\sys adaptively reduces the number of tasks in that sequence that it will
coalesce, committing sooner in future executions. Consequently, \sys's
coalescing mechanism allows a program to execute efficiently across a range of
energy buffer sizes, avoiding transition overheads in larger buffers, and
ensuring progress in smaller buffers. \TODO{articulate this last point better}

We evaluated our system on a collection of benchmark programs taken from the
literature, and compare directly to prior work, showing that \sys has high
performance and can flexible target a variety of platforms without recompiling
the code. We further demonstrate the efficiency and versatility of \sys with
an end-to-end evaluation of a sensing and data processing application deployed
in a lab environment. \TODO{blah blah blah eval}.

\TODO{Bullet contribs if there's space}