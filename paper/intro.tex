Advances in processor efficiency along with the development of energy-harvesting power systems has created a new category of devices that require neither a battery nor a tethered power supply~\cite{prasad_comst_2014,lucia_snapl_2017,soyata_csm_2016}. These devices operate entirely using energy extracted from their environment, such as RF energy~\cite{rf_powered_computing_gollakota_2014}, photovoltaic components~\cite{margolies_infocom_2016,margolies_tosn_2016}, and vibration~\cite{gorlatova_sigmetrics_2014}. Incorporating compute, storage, sensing, and communication hardware~\cite{wisp5,moo}, such energy-independent devices are a promising underlying technology for applications in emerging areas such as Internet of Things (IoT)~\cite{ku_cst_2016}, in-body~\cite{nadeau_naturebio_2017} and on-body~\cite{bandodkar_electroanalysis_2015} medical systems, and energy-harvesting nano-satellites~\cite{kicksat}.

While promising, energy-harvesting devices present a unique design challenge in
that they operate only {\em intermittently} when energy is available~\cite{hicks_isca_2017,lucia_snapl_2017}. An
energy-harvesting device buffers energy from its environment, e.g., in a
capacitor~\cite[Fig. 3]{gorlatova_tmc_2013},~\cite[Fig. 1]{gunduz_commag_2014}. When a threshold quantity of energy accumulates, the device begins operating. Harvestable energy sources typically produce relatively very little power compared to a platform's operating power level---compare for instance~\cite[Table III and V]{prasad_comst_2014} with~\cite[Table I]{carrano_cst_2014}. A devices operates for a brief period of time and when buffered energy is depleted, shuts down and begins recharging to operate again later. As an example, inter-task time may even exceed 20\,s for wireless energy-harvesting battery-less vivo digestive tract monitoring system~\cite[Fig. 3c]{nadeau_naturebio_2017}. On top of that, charge and discharge times vary by device and some of them~\cite{wisp} may fail $\approx$10 to $\approx$100 times per second~\cite[Fig. 1]{tan_infocom_2016},~\cite[Fig. 1]{mementos},~\cite[Fig. 3]{nvp}.

\todo{Put reference for "tictpl" somewhere}{Brandon}

Software in an energy-harvesting system operates according to the {\em
intermittent execution model}~\cite[Sec. 2]{dino}~\cite[Fig. 2]{lucia_snapl_2017}, with bursts of execution interrupted by failure periods. When power fails, a device loses its volatile state, i.e., registers, stack, SRAM, and retains its non-volatile state, i.e., FRAM, Flash. While capturing periodic checkpoints~\cite{mementos,quickrecall} and sleep scheduling~\cite{dewdrop,hibernus,hibernusplusplus} help to preserve execution progress, failures can leave volatile and non-volatile state inconsistent, leading to unrecoverable failures~\cite{mspcdino,edb}. 

%FRAM costs more than FRAM -- need to figure out how to execute out of SRAM

There are two fundamental approaches for dealing with data inconsistency for intermittently-powered devices: (i) programming and execution models~\cite{dino,ratchet,chain,alpaca} and (ii) architecture mechanisms~\cite{hicks_isca_2017,idetic,nvp}. Nonetheless, they both posses critical limitations. New architectures require costly hardware changes and are inapplicable to today's embedded off-the-shelf systems. For instance, they require dedicated idempotency violation control hardware buffers~\cite[Fig. 3]{hicks_isca_2017} or dedicated on-processor logic like on demand register backup~\cite[Fig. 9]{nvp}. Then, new programming models and compilers force programmers to use new memory models~\cite{chain,ratchet}, which may inhibit widespread adoption by programmers accustomed to classical (C) program writing. New memory models may also increase the amount of non-volatile memory accesses by requiring non-volatile versioning~\cite{chain} or completely precluding the use of volatile memory~\cite{ratchet}. Increasing pressure on non-volatile memory or precluding the use of volatile memory may degrade efficiency; e.g., FRAM has a higher access energy and latency than SRAM~\cite[Fig. 4]{nvp}---which will be also shown quantitatively in this paper. Commercially available devices are often provisioned with much less SRAM than FRAM, e.g. up to 64 FRAM/SRAM ratio of~\cite{wolverine}, and effectively leveraging the relatively high efficiency of SRAM is a key challenge for task-based intermittent systems. Despite the limitations, we conjecture that programming and execution models will be a {\em de facto choice for sustaining data consistency} of intermittently-powered devices. Simply put, programming model does not require any hardware investment and is dependent only upon the programmer's skills.

%Task sizing is hard -- how to chop without risking inefficiency or non-term

Two recent works~\cite{alpaca,chain} proposed {\em task-based} programming and execution paradigm, that require a programmer to statically decompose a program into tasks. A task resembles a function with no callers. Task can include arbitrary computation and, upon completion, is guaranteed to have executed {\em atomically} and to have logically {\em committed} memory updates. The programmer expresses the flow of control from one task to another and task control-flow may be input-dependent~\cite[Fig. 4]{chain}. An important programming challenge in task-based models is that the length of execution of a task is limited by the total amount of energy that a device can buffer and {\em does not change after compilation}. Assuming that input power is negligible compared to operating power, a task will never be able to complete if its execution consumes more energy than the system can buffer. On the other hand, transitioning from one task to the next imposes a cost and excessively small tasks may be inefficient. 

This brings the following research question: {\bf how to approach the appropriate task size to ensure progress and high performance given an unknown energy buffer size and energy arrival process?} More specifically, we ask two questions: (i) how to eliminate the need for hardware support while enabling the intermittent execution model to dynamically adapt task size {\em after compilation}, and (ii) how to reduce the energy consumption caused by frequent memory swapping operations during task traversals?

To answer these, in this work we develop {\bf \sys}: a new programming and execution model that utilizes two new runtime features: {\em dynamic task coalescing} to efficiently execute tasks without exceeding available energy (answering question (i)) and  {\em volatile memory virtualization} to leverage efficient volatile memory (answering question (ii)). \sys's novel memory virtualization mechanism uses a device's SRAM as the working memory during a task's execution, populating the SRAM with a page of data from FRAM as necessary at the start of the task. At the end of the task, the SRAM is committed back to FRAM using efficient DMA block copies~\cite{kistler_micro_2006}. When a task accesses data outside of a page in SRAM, the task must fetch data from FRAM. 

\todo{articulate this last point better}{Przemek}

With this paper we provide the following list of contributions.
%
\begin{enumerate}
%
\item {\bf Task coalescing mechanism:} \sys's core contribution is the introduction of dynamic task coalescing mechanism. It allows the programmer or a compiler to intuitively decompose the program into small tasks that amortize fixed per-task overheads, yet present no risk of exceeding device energy capacity. As such a decomposition executes, \sys~{\em coalesces dynamically} consecutive tasks. Coalescing elides the commits of coalesced tasks by buffering multiple tasks' updates in an FRAM commit buffer. Periodically, as the span of the coalesced task grows, \sys ends coalescing and commits the state of the coalesced task. If a power failure interrupts a sequence of coalesced tasks, \sys adaptively reduces the number of tasks in that sequence that it will coalesce, committing sooner in future executions. Consequently, \sys's coalescing mechanism allows a program to execute efficiently across a range of energy buffer sizes, avoiding transition overheads in larger buffers, and ensuring progress in smaller buffers.
%
\todo{Provide numerical results for this section}{Przemek}
\todo{Mention automatic task generation}{Przemek}
%
\item {\bf Evaluation of data swapping strategies:} We propose and explore two data swapping strategies for intermittently-powered systems based on FRAM/SRAM architecture. The first strategy is {\em demand paging}, which swaps the SRAM page with a new page from FRAM, buffering the swapped-out page until commit. The second strategy is {\em buffered direct access}, which directly reads and writes FRAM relying on dynamic double-buffering to ensure memory is consistent at
commit. 
%
\todo{Once paging is fully implemented--update the paragraph}{Przemek}
\todo{Provide numerical results for this section}{Przemek}
%
\item {\bf Extensive \sys evaluation:} We evaluated \sys on a collection of benchmark programs~\cite[Sec. 5]{chain}, and compare directly to state-of-the art task-based runtime~\cite{chain}, showing that \sys has high performance and can flexible target a variety of platforms without recompiling the code. We further demonstrate the efficiency and versatility of \sys with an end-to-end evaluation of a sensing and data processing application deployed in a lab environment. 
%
\todo{add detailed text about evaluation results}{Przemek}
%
\end{enumerate}